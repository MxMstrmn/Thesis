\section{Herleitung der Wanniergleichung}
\section{Lösen der Wanniergleichung}
Die Wanniergleichung beschreibt die als Exzitonen bezeichneten gebundenen Zustände zwischen einem Elektron im Leitungsband und dem im Valenzband existierenden Loch. Die Wanniergleichung ist eine direkte Folge aus der Coulombwechselwirkung und hat die Gestalt der 2-Teilchen-Schrödingergleichung, ist also vergleichbar mit einem effektiven Wasserstoffproblem. Aufgrund der deutlich geringeren effektiven Lochmasse $m_\nu$ gegenüber der Protonenmasse ist es notwendig das Problem in Schwerpunktkoordinaten zu transformieren und dann zu separieren. Für die Schwerpunktskoordinate ergibt sich die Bewegung eines freien Teilchens, wobei für die Relativkoordinate zwischen Elektron und Loch Folgendes gilt: 
\begin{alignat}{2}
  \mu &=\dfrac{m_\nu m_l}{m_\nu + m_l} && \qquad\text{Effektive Masse}\\
 \left [-\frac{\hbar^2}{2\mu}\nabla ^2 + V(r)\right ]\Psi(\vec{r}) &= E\, \Psi(\vec{r}) && \qquad\text{Wanniergleichung}\\
 V(r)&=-\dfrac{e^2}{4\pi\epsilon_0}\cdot \dfrac{1}{r} &&\qquad \text{Coulombpotential}
\end{alignat}
Grundsätzlich ist zu unterscheiden in wie vielen Dimensionen das Problem behandelt wird. Für 3 und 2 Dimensionen bleibt das Vorgehen allerdings das identisch, Unterschiede entstehen nur dadurch, dass im 2 dimensionalen Fall ein ausschließlich numerisch zu lösendes Integral auftaucht.\\ 
Der Lösungsweg führt hierbei über die Transformation der SGL in den Impulsraum, wobei das Faltungstheorem zur Bestimmung von $\mathcal{F}[V(r)\Psi(\Vr)](k)$ ausgenutzt wird. Dies ist möglich, da die $k$-Werte im reziproken Raum so dicht liegen, dass der Übergang von eigentlich diskreten $k$-Werten hin zur Integraldarstellung unter Berücksichtigung des k-Volumens $\frac{(2\pi)^d}{V}$ möglich ist. Darüber hinaus ist zu beachten, dass in Hinsicht auf die numerische Behandlung des Problems konvergenzerzeugende Faktoren notwendig sind, um das Problem in ein im Computer lösbares Eigenwertproblem zu überführen. \\ \\
Nach dieser groben Beschreibung soll das Problem nun in aller Ausführlichkeit besprochen werden. Es gilt: 
\begin{alignat*}{4}
 \Psi(\Vk)\coloneqq \mathcal{F}[\Psi(\Vr)](\Vk) && = \frac{1}{V}\int \Psi(\Vr)\e ^{\I\Vk\Vr}d^n r \qquad\qquad n=2,3 &&
 \mathcal{F}[\Delta \Psi(\Vr)](\Vk) && = -k^2\Psi(\Vk) \qquad \text{wegen Isotropie:}\qquad \Psi(\Vk)=\Psi(k)\\
 \widetilde{V}(k)\coloneqq \mathcal{F}[V(r)](k) && = \frac{1}{V}\int V(r)\e ^{\I\Vk\Vr}d^n r  &&
  \mathcal{F}[V(r)\Psi(\Vr)](\Vk) 	&& =\widetilde{V}(k)*\Psi(\Vk) \ \\
 && && \widetilde{V}(k)*\Psi(\Vk)  			&& = \sum\limits_{k'}  \widetilde{V}(k-k')\Psi(\Vk ') \\
 && && 														&& = \frac{V}{(2\pi)^n}\int  \widetilde{V}(k-k')\Psi(\Vk ') d^n k'
\end{alignat*}
\subsection{Lösen des 3D Problems}
Da sich das Problem in 3 Dimensionen sogar analytisch behandeln lässt, wird dieses zuerst behandelt. Für die Schrödingergleichung ergibt sich nun: 
\begin{align*}
\left [-\frac{\hbar^2}{2\mu}\nabla ^2 + V(r)\right ]\Psi(\vec{r}) &= E\, \Psi(\vec{r}) \qquad \leftrightarrow \qquad\left [\frac{\hbar^2k^2}{2\mu}-E  \right ]\Psi(k) =\underbrace{ \frac{V}{(2\pi)^3} \int  \widetilde{V}(k-k')\Psi(\Vk ') d^3 k'}_{= \, \text{Glg.1}} 
\end{align*}
Die Aufgabe ist nun Glg.1 zu lösen. Dabei gilt für die Fouriertransformierte des Coulombpotentials: 
\begin{alignat*}{2}
									&& &\widetilde V(k) = -\frac{1}{V}\frac{e^2}{\epsilon_0 |\Vk |^2} \qquad , \\
\text{sodass}\qquad 	&& \frac{V}{(2\pi)^3} \int  &\widetilde{V}(k-k')\Psi(k ') d^3 k' = \underbrace{-\frac{e^2}{(2\pi)^3 \epsilon_0}}_{\coloneqq c} \int  \Psi(k ')\frac{1}{|\Vk - \Vk '|^2} d^3 k'
\end{alignat*}
Mit Hilfe des Kosinussatzes und der Wahlfreiheit des Koordinatensystems ergibt sich: 
\begin{alignat*}{2}
									&& &|\Vk - \Vk ' |^2 = \Vk ^2 + \Vk '^2 -2|\Vk ||\Vk '|\cos(\vartheta) 	 \\
\text{folglich}\qquad 	&&  &\frac{e^2}{(2\pi)^3 \epsilon_0} \int  \Psi(k ')\frac{1}{|\Vk - \Vk '|^2} d^3 k' = -\dfrac{e^2}{4\pi^2 \epsilon_0} \int\limits_{0}^{\infty} \Psi(k ')\underbrace{\frac{k '}{k}\ln{\left |\frac{k + k '}{k-k'}\right |}}_{k\neq0\  \wedge \  k\neq k'} \, dk' 
\end{alignat*}

\begin{comment}
%
%		Diese Rechnungen gehören in die obige Gleichung uns sind die Lösung schritt für Schritt
%
\text{also}\qquad 		&& c \int  \Psi(k ')\frac{1}{|\Vk - \Vk '|^2} d^3 k' 	&= c\int\limits_{0}^{\infty}\int\limits_{0}^{\pi}\int\limits_{0}^{2\pi} \Psi(k ')\frac{k '^2 \sin(\vartheta)}{k ^2 + k '^2 -2kk '\cos(\vartheta)} \,d\varphi \, d\vartheta \, dk' \\
									&& &= 2\pi c\int\limits_{0}^{\infty}\int\limits_{-1}^{1} \Psi(k ')\frac{k '^2}{k ^2 + k '^2 -2kk 'u} \,du \, dk' \\
									&& &= -2\pi c\int\limits_{0}^{\infty} \Psi(k ')\frac{k '^2}{2kk '}\ln{(k ^2 + k '^2 -2kk 'u)}\Big |_{\text{-}1}^1 \,dk' \\
									&& &= \underbrace{2\pi \,c}_{-\dfrac{e^2}{4\pi^2 \epsilon_0}} \int\limits_{0}^{\infty} \Psi(k ')\frac{k '}{k}\ln{\left |\frac{k + k '}{k-k'}\right |} \, dk' \\
\end{comment}

An dieser Stelle ergeben sich zwei Schwierigkeiten. Zum einen wird das Integral für $k = 0$ singulär, was sich jedoch als harmlos erweist, wenn man den Grenzwert $k \rightarrow 0$ betrachtet. Zum anderen divergiert der Logarithmus im Integral $\forall k=k'$, was einen zunächst vor eine Herausforderung stellt. \\ \\
\subsubsection{Punkt 1: $k\rightarrow0$}
\begin{alignat*}{3}
\lim_{k\rightarrow 0}\,\frac{k '}{k}\ln{\left |\frac{k + k '}{k-k'}\right |} \ \  &&  \stackrel{\text{L.H.}}{=} \ \ \lim_{k\rightarrow 0}\,\frac{k '}{1}\frac{1}{\left |\frac{k + k '}{k-k'}\right |}\left |-\frac{2k'}{(k-k')^2}\right | \ \ = \ \  \lim_{k\rightarrow 0}\,\left |\frac{2k '^2}{{k^2 - k '^2}}\right | \ \ = \ \ 2 
\end{alignat*}
Damit ist gezeigt, dass der Integrand für $k \rightarrow 0$ einen endlichen Wert annimmt und damit nicht zum Integral beiträgt. 
\subsubsection{Punkt 2: $k = k'$}
Die Lösung dieses Problems teilt sich auf in zwei wesentliche Überlegungen. Zum einen ist zweckmäßig das Integral so aufzuteilen, dass sich ein Teil ergibt, der nur für $k\neq k'$ Beiträge liefert. Entsprechend gibt es einen anderen Teil, der sämtliche Beiträge enthält, bei denen $k=k'$ gilt. Die zweite Überlegung folgt aus der Konvergenzbetrachtung der Integrale. Da diese über den gesamten Radialbereich ausgewertet werden, kann durch Verändern des Konvergenzverhaltens des Integranden der Grenzwert eines eigentlich divergierenden Integrals bestimmt werden. \\
Konkret ergibt sich: 
\begin{alignat*}{2}
\int\limits_{0}^{\infty} \Psi(k ')\underbrace{\frac{k '}{k}\ln{\left |\frac{k + k '}{k-k'}\right |}}_{\coloneqq \, V_\text{eff}(k,k')}\, dk' && = \int\limits_{0}^{\infty} \underbrace{\Psi(k ')\frac{k '}{k}\ln{\left |\frac{k + k '}{k-k'}\right |}-X}_{\stackrel{!}{=}0,\text{falls }k=k',\text{ Teil 1}}\, dk' + 
		 	\int\limits_{0}^{\infty} \underbrace{\Psi(k ')\frac{k '}{k}\ln{\left |\frac{k + k '}{k-k'}\right |}X}_{\text{Teil 2}}\, dk' 
\end{alignat*}
Es ist sofort ersichtlich, dass $X \propto \Psi(k)$ sein sollte, um die an Teil 1 gestellte Bedingung zu erfüllen. Bezüglich der Lösbarkeit von Teil 2 muss außerdem ein Konvergenzfaktor berücksichtigt werden, der die Form $g(k,k') = \frac{4k^4}{(k^2 + k'^2)^2}$ besitzt. So ergibt sich insgesamt für $X$: 
\begin{alignat*}{4}
&& X = g(k,k')\frac{V_\text{eff}(k,k')}{\Psi(k')}\Psi(k)  &&\ \ =\ \ && &\frac{4k^4}{(k^2+k '^2) ^2}\frac{V_\text{eff}(k,k')}{\Psi(k')}\Psi(k) \\
\text{Teil 1:}\qquad 	&&  \int\limits_{0}^{\infty} \Psi(k ')V_\text{eff}(k,k')-X \,dk' && \ \ = \ \ &&
											&\int\limits_{0}^{\infty} V_\text{eff}(k,k')\left [\Psi(k ')-\frac{4k^4}{(k^2+k '^2)^2}\Psi(k)\right ]\,dk'  \\ 
\text{Teil 2:}\qquad 	&& 	\int\limits_{0}^{\infty} \Psi(k ')V_\text{eff}(k,k')X   && \ \ = \ \ &&
											&\Psi(k)\int\limits_{0}^{\infty} \frac{4k^4}{(k^2+k '^2)^2}V_\text{eff}(k,k')\,dk'
\end{alignat*}
In Hinblick auf die numerische Behandlung muss Teil 2 noch weiter analysiert werden, während Teil 1 schon diskretisiert werden kann. Dazu löst man den Betrag im Logarithmus auf und integriert partiell, sodass sich ein einfaches Integral bildet, welches sich aufteilt in Logarithmen und einen Arkustangens.
\begin{alignat*}{2}
\int\limits_{0}^{\infty} \frac{4k^4}{(k^2+k '^2)^2}V_\text{eff}(k,k')\,dk' 
						&& \ \ &=\ \ \int\limits_{0}^{\infty} \frac{4k^4}{(k^2+k '^2)^2}\frac{k '}{k}\ln{\left |\frac{k + k '}{k-k'}\right |}\,dk' \\
%
%						Hier kommen Gleichungen aus untenstehendem Kommmentar rein
%
						&& \ \ &=\ \ 	k\Big[\underbrace{\ln\left (\frac{1+u}{1-u}\right )}_{\rightarrow 0 \text{ für } u\rightarrow \infty}+\, 2\arctan{u}\Big]_{0}^{\infty}\\  
						&& \ \ &=\ \ 	\pi k
\end{alignat*}
\begin{comment}
%
%					Diese Zeilen sind die explizite Rechnung des Integrals oben
%
						&& \ \ &=\ \ 	k\int\limits_{0}^{\infty} \frac{4u}{(1+u^2)^2}\ln{\left |\frac{1 + u}{1-u}\right |}\,du \qquad\text{Betrag auflösen}\\
						&& \ \ &=\ \ 	k\int\limits_{0}^{1} \left (\frac{-2}{1+u^2}\right )'\ln{\left (\frac{1 + u}{1-u}\right )}\,du + k\int\limits_{1}^{\infty} \left( \frac{-2}{1+u^2}\right )'\ln{\left (\frac{1 + u}{u-1}\right )}\,du\\
						&& \ \ &\stackrel{\text{P.I.}}{=}\ \ \underbrace{k\left (\frac{-2}{1+u^2}\right )\ln{\left ( \frac{1 + u}{1-u}\right )}\Big |_{0}^{1}+k\left (\frac{-2}{1+u^2}\right )\ln{\left ( \frac{1 + u}{u-1}\right )}\Big |_{1}^{\infty}}_{\text{C.H.}\,\rightarrow \,0}
						+k\int\limits_{0}^{\infty} \frac{2}{1+u^2}\frac{1 - u}{1+u}\frac{2}{(1-u)^2}\,du \\
						&& \ \ &=\ \ 	k\int\limits_{0}^{\infty} \frac{4}{(1-u^4)}\,du  \\
						&& \ \ &=\ \ 	k\int\limits_{0}^{\infty} \left (\frac{1}{1-u}+\frac{1}{1+u}+\frac{2}{1+u^2}\right )\,du\\
						&& \ \ &=\ \ 	k\Big[-\ln{(1-u)}+\ln{(1+u)}+ 2\arctan{u}\Big]_{0}^{\infty}\\  
%
\end{comment}

Damit ergibt sich nun: 
\begin{alignat*}{1}
	&-\dfrac{e^2}{4\pi ^2 \epsilon_0} \Big [\int\limits_{0}^{\infty} V_\text{eff}(k,k')\left [\Psi(k ')-g(k,k')\Psi(k)\right ]\,dk'  
+  \Psi(k)\int\limits_{0}^{\infty} g(k,k')V_\text{eff}(k,k')\,dk' \Big]   \\
= 	\qquad &-\dfrac{e^2}{4\pi^2 \epsilon_0} \int\limits_{0}^{\infty} V_\text{eff}(k,k')\left [\Psi(k ')-g(k,k')\Psi(k)\right ]\,dk' 
-  \dfrac{e^2}{4\pi \epsilon_0}  k\Psi(k) 
\end{alignat*}
Die Diskretisierung des Problems hat somit die Form: 
\begin{alignat*}{3}
\sum\limits_{k'}H_{kk'}\Psi_{k'} = E \Psi_k && \qquad \text{  wobei}\qquad && &H_{kk'}= 
	 \begin{cases}
     V_{kk'}  &k \neq k' \\
     \frac{\hbar ^2 k^2}{2\mu} -\frac{e^2}{4\pi \epsilon_0}k +\sum\limits_{k\neq k'}\frac{4k^4}{(k^2+k '^2)^2}V_{kk'} &k=k'
   \end{cases}\\
  &&  && &V_{kk'} = V_\text{eff}(k,k')=\frac{k '}{k}\ln{\left |\frac{k + k '}{k-k'}\right |}
\end{alignat*}
\subsection{Lösen des 2D Problems}
Das Vorgehen beim 2D Problem ist völlig analog zur Lösung in 3 Dimensionen. Einzig unterschiedlich ist die Lösbarkeit der Integrale, welche nur numerisch ausgewertet werden können. Angefangen wird erneut mit der Fouriertransformation des Coulombpotentials, welche die folgende Form besitzt: 
\begin{alignat*}{1} 
\mathcal{F}[V(r)](k)=-\frac{e^2}{2\varepsilon_0 V}\frac{1}{|\Vk|}\eqqcolon \widetilde{V}(k)
\end{alignat*}
Natürlich gilt für die Fouriertransformation der Schrödigergleichung eines isotropen Problems: 
\begin{alignat*}{2}
\left [-\frac{\hbar^2}{2\mu}\nabla ^2 + V(r)\right ]\Psi(\vec{r}) &= E\, \Psi(\vec{r}) \ \ \leftrightarrow \ \ \left [\frac{\hbar^2k^2}{2\mu}-E  \right ]\Psi(k) = \frac{V}{4\pi ^2} \int  \widetilde{V}(k-k')\Psi(k ') d^2 k'
\end{alignat*}
Das Faltungsintegral hat hierbei die Form: 
\begin{alignat*}{2}
\frac{V}{4\pi ^2} \int  \widetilde{V}(k-k')\Psi(\Vk ') d^2 k' 
		&& \ \ &= \ \ -\frac{e^2}{8\pi^2\varepsilon_0} \int  \frac{1}{|\Vk -\Vk '|}\Psi(k ') d^2 k' \\
		&& \ \ &= \ \ -\frac{e^2}{8\pi^2\varepsilon_0} \int\limits_{0}^{\infty}\int\limits_{0}^{2\pi}  \frac{k'}{\sqrt{k^2+k'^2-2kk'\cos(\varphi)}}\Psi(k ') d\varphi dk' \\
		&& \ \ &= \ \ \int\limits_{0}^{\infty}V_\text{eff}(k,k')\Psi(k ') dk' \\
\text{wobei}\qquad V_\text{eff}(k,k') &&  \ \ &= \ \ 		-\frac{e^2}{8\pi^2\varepsilon_0}\int\limits_{0}^{2\pi}  \frac{k'}{\sqrt{k^2+k'^2-2kk'\cos(\varphi)}} d\varphi
\end{alignat*}
Da das Integral über die Winkelkoordinate nun nicht mehr trivial zu lösen ist, hängt das effektive Potential $V_\text{eff}$ noch direkt von diesem ab. Wieder ergeben sich Schwierigkeiten an den Stellen, an denen der Integrand divergiert. Dies ist insbesondere der Fall, wenn $k=k'$ und $\varphi = 0$ gilt. Um dieses Problem zu lösen, wird der Faltungsausdruck erneut mithilfe des konvergenzerzeugenden Faktors $g(k,k')$  umgeschrieben: 
\begin{alignat*}{1}
g(k,k') \ \ &= \ \ \frac{4k^4}{(k^2+k '^2)^2} \\ 
\int\limits_{0}^{\infty}V_\text{eff}(k,k')\Psi(k ') dk' 
&=\underbrace{ \int\limits_{0}^{\infty}V_\text{eff}(k,k')\left [\Psi(k ')-g(k,k')\Psi(k)\right ] dk'}_{\text{Teil 1}} +\Psi(k) 
\underbrace{\int\limits_{0}^{\infty}g(k,k')V_\text{eff}(k,k')dk'}_{\text{Teil 2}}
\end{alignat*}
Es zeigt sich, dass der Integrand in Teil 2 dimensionslos dargestellt werden kann, woraus sich der Vorteil ergibt, dass dieses Integral unabhängig von $k$ oder $k'$ auf einem feinen $\varphi$-Gitter bestimmt werden kann. Es gilt: 
\begin{alignat*}{2}
\int\limits_{0}^{\infty}g(k,k')V_\text{eff}(k,k')dk' 
		&& \ \ &= \ \ \underbrace{-\frac{e^2}{8\pi^2\varepsilon_0}}_{\coloneqq \,c} 
		\int\limits_{0}^{\infty}\frac{4k^4}{(k^2+k '^2)^2}\int\limits_{0}^{2\pi}  \frac{k'}{\sqrt{k^2+k'^2-2kk'\cos(\varphi)}} d\varphi dk' \\
		&&\ \ &= \ \ c\cdot k \int\limits_{0}^{\infty}\int\limits_{0}^{2\pi} \frac{4u}{(1+u^2)^2\sqrt{1+u^2-2u\cos(\varphi)}} d\varphi du\\ 
		&&\ \ &= \ \ -\frac{e^2}{8\pi^2\varepsilon_0} k I
\end{alignat*}
Da $I$ konvergiert und damit numerisch zu bestimmen ist, folgt für die Diskretisierung des Problems die identische Darstellung wie in 3 Dimensionen:
\begin{alignat*}{3}
\sum\limits_{k'}H_{kk'}\Psi_{k'} = E \Psi_k && \qquad \text{  wobei}\qquad && &H_{kk'}= 
	 \begin{cases}
     V_{kk'}  &k \neq k' \\
     \frac{\hbar ^2 k^2}{2\mu} -\frac{e^2}{8\pi^2\varepsilon_0} k I +\sum\limits_{k\neq k'}\frac{4k^4}{(k^2+k '^2)^2}V_{kk'} &k=k'
   \end{cases}\\
  &&  && &V_{kk'} = V_\text{eff}(k,k')=-\frac{e^2}{8\pi^2\varepsilon_0}\int\limits_{0}^{2\pi}  \frac{k'}{\sqrt{k^2+k'^2-2kk'\cos(\varphi)}} d\varphi
\end{alignat*}
