
% \documentclass[CEJM,DVI]{cej} % use DVI command to enable LaTeX driver
\documentclass[CEJM,PDF]{cej} % use PDF command to enable PDFLaTeX driver
\usepackage{layout}
\usepackage{amsmath}
\usepackage{mathtools}
\usepackage{mwe}    % loads »blindtext« and »graphicx«
\usepackage{subfig}
\usepackage{float}


\def\lorem{}

\def\LOREM{\lorem \\ \lorem}

\usepackage[utf8]{inputenc}
\usepackage[ngerman]{babel}
\usepackage[T1]{fontenc}
\usepackage[svgnames]{xcolor}
\definecolor{cool}{HTML}{0063a4}

\usepackage{hyperref}% http://ctan.org/pkg/hyperref
\hypersetup{%
  colorlinks = true,
  linkcolor  = cool
}




\newcommand{\mos}{MoS$_2$ }
\newcommand{\mose}{MoSe$_2$ }
\newcommand{\ws}{WS$_2$ }
\newcommand{\Vr}{\vec{r} }
\newcommand{\Vk}{\vec{k} }
\newcommand{\e}{\text{e} }
\newcommand{\I}{\text{i} }

%\newcommand*{\rom}[1]{\expandafter\@slowromancap\romannumeral #1@}\\

\title{Wissenswertes zu TMDC und allgemein 2D Materialien}

%\articletype{Article category} % Research Article, Review Article, Communication, Erratum

\articletype{}

\author{Leon Hetzel\email{lhetzel@uni-bremen.de}
       }

\shortauthor{L. Hetzel}

\institute{
           Department of Physics, University of Bremen }

\abstract{Das folgende ist ein Zusammenschrieb bezüglich der wichtigsten Aspekte zu 2D Materialien und im Besonderen der Klasse der TMDCs. 
}

\keywords{}

\msc{}

\begin{document}
\maketitle
%\baselinestretch{2}
\section{Übergangsmetall Dichalkogenide}
\subsection*{Herstellung}
Für die wissenschaftliche Untersuchung der elektrischen und optischen Eigenschaften von 2D-Materialien ist es unabdingbar zuverlässig atomar dünne Strukturen mit einheitlichen Eigenschaften herstellen zu können. In diesem Abschnitt sollen die zu den zwei prinzipiell verschiedenen Ansätzen,  "top-down"\ und "bottom-up", gehörenden Methoden vorgestellt werden. Wie die Bezeichnung schon andeutet, handelt es bei den top-down Methoden um jene, die von einem ganzen Kristall ausgehen, um dünne Schichten herzustellen, und bei den bottom-down Methode um jene, die dünne Schichten des Materials auf einem Substrat anwachsen lassen. Vorzüge und Nachteile der verschiedenen Methoden lassen sich anhand der wichtigsten Kenngrößen ablesen: Reinheit, Dicke und flächenmäßige Ausdehnung. In Hinblick auf mögliche Anwendungen spielt insbesondere die Skalierbarkeit der Methoden eine Rolle. 

\subsubsection*{Top-down Methoden}
Um atomar dünne Schichten von TMDCs herzustellen, lassen sich grundsätzlich dieselben Methoden anwenden, die für Graphen entwickelt wurden[]. Man nutzt Klebeband, um dünne Schichten des Materials von seiner Kristallstruktur mikromechanisch zu lösen. Auf diese Weise erhält man sehr reine Proben, die sich zur fundamentalen Untersuchung sowie individuelle Anwendungen eignen, jedoch aufgrund der mangelnden Skalierbarkeit der Methode nicht zur industriellen Nutzung. \\
Um große Mengen von Nanostrukturen zu erhalten, bietet sich die Flüssigphasen-Präparation an. Diese lässt über einfache Nanostrukturen auch die Herstellung von Kompositen und Hybriden (Unterschied?) zu, indem einfach Dispersionen unterschiedlicher Materialien vermischt werden. Weiterhin ist es bei dieser Methode vergleichsweise leicht, weiterführende chemische Verfahren anzuwenden, welche zu verschiedenen, lösungsorientierten Strukturen führen können. Die Erfolge, die hierdurch bei Graphen erreicht wurden, versprechen auch für die TMDCs ähnlich gute Aussichten.\\
Darüber hinaus eignet sich die Interkalation mittels Ionen, um auch in Flüssigkeiten die gewünschten dünnen Sichten abzulösen. Hierbei wird Puder eines TMDC in ein Lösemittel gegeben, welche außerdem versetzt ist mit beispielsweise Lithium-Ionen. Wenn die Ionen inkaliert sind, wird das Material in Wasser gegeben, wo die Lithium-Ionen stark reagieren und unter Freisetzung von H$_2$ Gas die einzelnen Materialschichten voneinander trennen. Zwar ist der Ertrag dieser Methode sehr hoch, doch wird durch diesen chemischen Eingriff die Struktur des ursprünglichen Kristalls zerstört, sodass sich die elektronischen Eigenschaften verändern. (Details möglich). Für MoS$_2$ kann die Atomkonfiguration durch Erhitzen auf 300$^\circ$C wieder auf die halbleitende Struktur zurückgeführt werden. Erweiternd kann die Lithiation kontrolliert werden, wenn eine elektrochemische Zelle genutzt wird, bei der Lithiumfolie als Anode und ein TMDC Material als Kathode dient und die über ihre Entladung den Grad der Lithiation bestimmt. Da diese Methode nur noch einige Stunden in Anspruch nimmt und nicht mehr mehr als einen ganzen Tag, kommt es außerdem zu einer Zeitersparnis.\\
Ähnlich hierzu lässt sich auch Ultraschall zusammen mit der passenden Wahl eines Lösemittels (Details?) nutzen, um Nanostrukturen weniger 100\,nm herzustellen. Dabei ist es wichtig die spontane Wiedervereinigung zu verhindern und festzustellen, ob sich durch die Anlagerung gelöster Moleküle and die Nanostrukturen elektronische Eigenschaften ändern. \\
Trotz der vergleichbar hohen Kosten für Lithium und der Gefahr der Entflammbarkeit ist die Methode der Lithium Interkalation sicher die passende in Hinblick auf elektronische und photonische Anwendungen, bei der Monolagen benötigt werden. Wenn jedoch große Mengen synthetisiert werden müssen und die genaue Dicke der Proben eine untergeordnete Rolle spielt, dann ist die Flüssigphasen-Präparation vorzuziehen. Dennoch ist noch viel Arbeit und Forschung notwendig, um die notwendige Qualität der Proben mit einem entsprechend großen Ertrag vereinen zu können. 

\subsubsection*{Bottom-up Methoden}
Insbesondere für die Nutzung in Anwendungen, wie (wafer-scale?)  elektronsichen oder flexiblen, transparenten optoelektronischen Geräten, ist es ein wichtiger Schritt, flächenmäßig große und einheitliche Strukturen herstellen zu können. Für Graphen ermöglichte die Entwicklung von Methoden chemischen Gesphasenabscheidung (CVD) die industrielle Produktion von Anwedungen/Geräten. \\ 
Ziel der chemischen Gasphasenabscheidung ist die Abscheidung des gewünschten Feststoffes auf einem Subsrat. Dafür ist es notwendig, dass flüchtige Verbindungen der Schichtkomponenten existieren und diese bei einer bestimmten Temperatur die feste Schicht bilden und sich auf dem Substrat anlagern. Eine Beschränkung bildet hierbei die hohe Temperaturbelastung der Substrate. Für die Synthetisierung von \mos konnten schon erste Erfolge mit CVD Methoden verzeichnet werden. Abhängig vom verwendeten Präkursor wurden verschiede Schichtdicken von \mos erreicht, die vollständige Kontrolle über die genaue Lagenanzahl konnte jedoch noch nicht erreicht werden[]. \\
Ebenso konnten hydrothermale Züchtungsverfahren genutzt werden, um \mos und \mose zu synthetisieren. Das heißt, der Kristall wird aus einer wässrigen Lösung in einem Autoklav bei hohen Temperaturen und hohem Druck gezüchtet. 


\subsection*{Eigenschaften}
\subsubsection*{Elektrische Eigenschaften}
TMDC Materialien besitzen häufig, wie zum Beispiel durch spektroskopische Untersuchungen oder Tight Binding Berechnungen gezeigt, ähnliche Eigenschaften. Man unterscheidet grundsätzlich zwischen solchen TMDCs, die metallisches Verhalten zeigen, und solchen, die halbleitend sind. Darüber hinaus gibt es auch solche, die mitunter exotische Eigenschaften besitzen, beispielsweise supraleitend sind oder Ladungsdichtewellen ausbilden. Die Bandstruktur aller TMDC Materialien ändert sich mit der Anzahl der Lagen des Materials. So unterliegen halbleitende TMDCs einem Übergang von einer indirekten Bandlücke hin zu einer direkten Bandlücke für eine Monolagen-Struktur. Dabei bleiben die direkten exzitonischen Übergänge von \mos oder \ws am K-Punkt weitestgehend unverändert, jedoch ändert sich die Bandstruktur signifikant in der Nähe des $\Gamma$-Punktes aufgrund der räumlichen Einschränkung in einer Dimension (quantum condement) und der  resultierenden Veräderung der Hybridisierung. Dieses Verhalten erwartet man bei allen Materialien der Form MoX$_2$ und WX$_2$. Da die Bandlücken halbleitender TMDCs vergleichbar sind zu der in Silizium (1.1\,eV) erscheinen diese passend für den Gebraucht als Transistoren. Insbesondere der Übergang hin zur direkten Bandlücke hat enormen Einfluss auf die Tauglichkeit in Hinblick auf Photonik, Optoelektronik oder Sensoren. \\

Eine der wichtigsten Anwendungen von Halbleitern sind Transistoren. Stand der Technik für aktuelle Prozessoren sind die Silikon basierten MOSFETs, Metall-Oxid-Halbleiter-Feldeffektransistoren. Doch aufgrund von Wärmedissipation und Quanteneffekten, die in Nanometerdimensionen immer weiter zum Tragen kommen, ist die Skalierbarkeit begrenzt und es ist unwahrscheinlich Transistorlängen deutlich kleiner als 22\,nm zu erreichen. Nicht zuletzt wegen ihrer Größe stellen insbesondere 2D Transistoren eine vielversprechende Alternative zur etablierten Technologie dar. Tatsächlich konnte der grundsätzliche Aufbau eines Feldeffekt-Transistors bereits auf TMDC Materialien übertragen werden [Details FET?]. Zwar besitzt Graphen eine hohe Ladungsträger Beweglichkeit und konnte für den Bau von Transistoren [Art?] genutzt werden, jedoch verhindert die fehlende Bandlücke von Graphen einen geringen Reststrom und damit die Nutzbarkeit von Graphen in der digitalen Schaltungstechnik. Der Bedarf nach neuen Nanostrukturen, die über die bisherigen Möglichkeiten hinaus gehen, ist damit ungebrochen. \\ 

Erste Bemühungen haben gezeigt, dass TMDCs als channel Material in FET genutzt werden können und leistungsarme und platzsparender Elektronik mit \mos realisierbar ist. Begründet durch die bemerkenswerten mechanischen Eigenschaften von \mos, 30 mal so stark wie Stahl und eine Verformbarkeit von bis zu 11\%, ist \mos das beste Halbleitermaterial, um flexible Strukturen zu erstellen. So konnten flexible Transistoren schon realisiert werden und zusammen mit anderen 2D Materialien erscheinen hybride 2D-Strukturen auf flexiblen Substraten realistisch. Ebenso vielversprechend sind hybride Strukturen, die durch das aufeinander auftragen verschiedener 2D-Materialien entstehen und allein wegen ihrer neuartigen Architektur völlig neue Eigenschaften bereit halten. Zuletzt wurde ein sogenannter Tunnel-Feldeffekt-Transistor vorgestellt.\\

\subsubsection*{Optische Eigenschaften}
Die Veränderung der Bandstruktur von TMDC Materialien beeinflusst natürlich auch die optischen Eigenschaften des Materials wie zum Beispiel die Photoleitfähigkeit, das Absorbtionsspektrum oder die Photolumineszenz. Letztere vergrößert sich gegenüber der Kristallstruktur um einen Faktor von bis zu $10^4$. Der sogenannte quantum yield ist jedoch auch sehr vom Substrat abhängig, sodass sich Ergebnisse für unterschiedliche Substrate unterscheiden und insgesamt deutlich unter den für einen Halbleiter mit direkter Bandlücke erwarteten Werten liegen. Diese Beobachtungen zeigen, dass weiterer Forschung bedarf, um \mos und andere TMDCs zu verstehen und die Materialien so zu kontrollieren, dass der quantum yield größer wird. Für \mos Monolagen befindet entspricht der Hauptpeak der direkten Bandlücke, wogegen MoS$_2$, welches aus mehreren Lagen besteht, auch weitere durch die indirekte Bandlücke hervorgerufene Peaks zeigt. Die Kristallstruktur zeigt genau zwei Hauptpeaks, was mit den Exzitonenergien am K-Punkt übereinstimmt. Rechnungen konnten außerdem zeigen, dass die Bandaufaufspaltung direkt mit der Spin-Orbit-Wechselwirkung in der Monolage zusammenhängt. Darüber hinaus konnten die hohen Exziton Bindungsenergien in durch die größeren dielektrischen Konstanten für die Bi- und Monolagen-Struktur [Physik?] erklärt werden. \\

Wie zuvor schon angedeutet besteht ein großes Interesse an flexiblen und transparenten optoelektronischen Anwendungen. Denkbar wären neuartige Displays oder auch tragbare, elektronische Geräte im Allgemeinen (waerables). Für die Umsetzung dieser Ideen wird eine Vielzahl verschiedener transparenter und flexibler Komponenten benötigt, beispielsweise Halbleiter, Leiter, optische Absorber (optical absorber), Emitter oder dielektrische Materialien, die wiederum verschiedener Klassen von 2D Materialien bedürfen. Insbesondere für Licht absorbierende und emittierende Materialien scheinen die TMDCs sehr durch ihren halbleitende Charakter und die Möglichkeit der einstellbaren Bandlücke geeignet zu sein. Andere "D Materialien wie Perovskite und BN sind wichtiger in Hinblick auf die Realisierung von 2D dielektrischen Materialien. \\ 

Die Fähigkeit Licht zu absorbieren führt unweigerlich auf mögliche Anwendungen im Bereich von Photovoltaik. Schon heute sind die Austrittsarbeiten mit den üblichen Materialien vergleichbar. Darüber hinaus können durch die Verwendung von \mos verschiedener Materialdicken und daraus resultierenden unterschiedlichen Bandlücken verschiedene Wellenlängen absorbiert werden. Durch das Stapeln dieser 2D TMDC Materialien werden so Zellen denkbar, welche Photonen aus dem gesamten Sonnenspektrum absorbieren und Verluste durch Thermalisation vermeiden. \\ 

Ebenso ermöglicht die direkte Bandlücke die Herstellung von LEDs (light emitting diodes) welche außerdem die mechanischen Eigenschaften der TMDCs besitzen, also flexibel und transparent sind. Dies allein macht TMDCs zu perfekten Kandidaten zur Erforschung von zukünfitgen optoelektronischen Anwendungen. Hier jedoch besteht noch große Ungewissheit bzgl. der tatsächlichen Ausbeute, wie zuvor schon erwähnt, und es stehen noch viele ungeklärte Fragen im Raum. \\

\subsubsection*{Spintroniks und Valley Interaktionen}
Wie die Ladung der Elektronen benutzt wird, um nwendungen zu ermöglichen, so lässt sich auch die Eigenschaft einen Spin zu besitzen ausnutzen, um allgemein gesprochen Signale zu transportieren. Bei letzterem spricht man von Spintronik. Der Begriff Valleytronics bezieht sich nun auf die Möglichkeit Elektronen bzw. Löcher nicht bloß abhängig vom Spin sondern auch vom Impulsraum anzuregen. Die hierfür ursächliche Spin-Orbit-Wechselwirkung ist in TMDC Materialien sehr stark ausgeprägt. Dies liegt an der Brechung der Inversionssymmetry, der Einschränkung der Elektronen auf 2 Dimensionen und nicht zuletzt an den großen Massen der Atome in MX$_2$ Materialien. Die Aufspaltung des Valenzbands liegt dadurch im Bereich von 0,15 bis 0,45\,eV. Auch experimentell konnte diese Einschränkung der Täler in der Bandstruktur bestätigt werden, indem \mos mit zirkular polarisiertem Licht bestrahlt wurde und dadurch die Ladungsträgerpopulationen auf bestimmte Bereiche der Bandstruktur begrenzt werden konnte.
\section{Herleitung der Wanniergleichung}
\section{Lösen der Wanniergleichung}
Die Wanniergleichung beschreibt die als Exzitonen bezeichneten gebundenen Zustände zwischen einem Elektron im Leitungsband und dem im Valenzband existierenden Loch. Die Wanniergleichung ist eine direkte Folge aus der Coulombwechselwirkung und hat die Gestalt der 2-Teilchen-Schrödingergleichung, ist also vergleichbar mit einem effektiven Wasserstoffproblem. Aufgrund der deutlich geringeren effektiven Lochmasse $m_\nu$ gegenüber der Protonenmasse ist es notwendig das Problem in Schwerpunktkoordinaten zu transformieren und dann zu separieren. Für die Schwerpunktskoordinate ergibt sich die Bewegung eines freien Teilchens, wobei für die Relativkoordinate zwischen Elektron und Loch Folgendes gilt: 
\begin{alignat}{2}
  \mu &=\dfrac{m_\nu m_l}{m_\nu + m_l} && \qquad\text{Effektive Masse}\\
 \left [-\frac{\hbar^2}{2\mu}\nabla ^2 + V(r)\right ]\Psi(\vec{r}) &= E\, \Psi(\vec{r}) && \qquad\text{Wanniergleichung}\\
 V(r)&=-\dfrac{e^2}{4\pi\epsilon_0}\cdot \dfrac{1}{r} &&\qquad \text{Coulombpotential}
\end{alignat}
Grundsätzlich ist zu unterscheiden in wie vielen Dimensionen das Problem behandelt wird. Für 3 und 2 Dimensionen bleibt das Vorgehen allerdings das identisch, Unterschiede entstehen nur dadurch, dass im 2 dimensionalen Fall ein ausschließlich numerisch zu lösendes Integral auftaucht.\\ 
Der Lösungsweg führt hierbei über die Transformation der SGL in den Impulsraum, wobei das Faltungstheorem zur Bestimmung von $\mathcal{F}[V(r)\Psi(\Vr)](k)$ ausgenutzt wird. Dies ist möglich, da die $k$-Werte im reziproken Raum so dicht liegen, dass der Übergang von eigentlich diskreten $k$-Werten hin zur Integraldarstellung unter Berücksichtigung des k-Volumens $\frac{(2\pi)^d}{V}$ möglich ist. Darüber hinaus ist zu beachten, dass in Hinsicht auf die numerische Behandlung des Problems konvergenzerzeugende Faktoren notwendig sind, um das Problem in ein im Computer lösbares Eigenwertproblem zu überführen. \\ \\
Nach dieser groben Beschreibung soll das Problem nun in aller Ausführlichkeit besprochen werden. Es gilt: 
\begin{alignat*}{4}
 \Psi(\Vk)\coloneqq \mathcal{F}[\Psi(\Vr)](\Vk) && = \frac{1}{V}\int \Psi(\Vr)\e ^{\I\Vk\Vr}d^n r \qquad\qquad n=2,3 &&
 \mathcal{F}[\Delta \Psi(\Vr)](\Vk) && = -k^2\Psi(\Vk) \qquad \text{wegen Isotropie:}\qquad \Psi(\Vk)=\Psi(k)\\
 \widetilde{V}(k)\coloneqq \mathcal{F}[V(r)](k) && = \frac{1}{V}\int V(r)\e ^{\I\Vk\Vr}d^n r  &&
  \mathcal{F}[V(r)\Psi(\Vr)](\Vk) 	&& =\widetilde{V}(k)*\Psi(\Vk) \ \\
 && && \widetilde{V}(k)*\Psi(\Vk)  			&& = \sum\limits_{k'}  \widetilde{V}(k-k')\Psi(\Vk ') \\
 && && 														&& = \frac{V}{(2\pi)^n}\int  \widetilde{V}(k-k')\Psi(\Vk ') d^n k'
\end{alignat*}
\subsection{Lösen des 3D Problems}
Da sich das Problem in 3 Dimensionen sogar analytisch behandeln lässt, wird dieses zuerst behandelt. Für die Schrödingergleichung ergibt sich nun: 
\begin{align*}
\left [-\frac{\hbar^2}{2\mu}\nabla ^2 + V(r)\right ]\Psi(\vec{r}) &= E\, \Psi(\vec{r}) \qquad \leftrightarrow \qquad\left [\frac{\hbar^2k^2}{2\mu}-E  \right ]\Psi(k) =\underbrace{ \frac{V}{(2\pi)^3} \int  \widetilde{V}(k-k')\Psi(\Vk ') d^3 k'}_{= \, \text{Glg.1}} 
\end{align*}
Die Aufgabe ist nun Glg.1 zu lösen. Dabei gilt für die Fouriertransformierte des Coulombpotentials: 
\begin{alignat*}{2}
									&& &\widetilde V(k) = -\frac{1}{V}\frac{e^2}{\epsilon_0 |\Vk |^2} \qquad , \\
\text{sodass}\qquad 	&& \frac{V}{(2\pi)^3} \int  &\widetilde{V}(k-k')\Psi(k ') d^3 k' = \underbrace{-\frac{e^2}{(2\pi)^3 \epsilon_0}}_{\coloneqq c} \int  \Psi(k ')\frac{1}{|\Vk - \Vk '|^2} d^3 k'
\end{alignat*}
Mit Hilfe des Kosinussatzes und der Wahlfreiheit des Koordinatensystems ergibt sich: 
\begin{alignat*}{2}
									&& &|\Vk - \Vk ' |^2 = \Vk ^2 + \Vk '^2 -2|\Vk ||\Vk '|\cos(\vartheta) 	 \\
\text{folglich}\qquad 	&&  &\frac{e^2}{(2\pi)^3 \epsilon_0} \int  \Psi(k ')\frac{1}{|\Vk - \Vk '|^2} d^3 k' = -\dfrac{e^2}{4\pi^2 \epsilon_0} \int\limits_{0}^{\infty} \Psi(k ')\underbrace{\frac{k '}{k}\ln{\left |\frac{k + k '}{k-k'}\right |}}_{k\neq0\  \wedge \  k\neq k'} \, dk' 
\end{alignat*}

\begin{comment}
%
%		Diese Rechnungen gehören in die obige Gleichung uns sind die Lösung schritt für Schritt
%
\text{also}\qquad 		&& c \int  \Psi(k ')\frac{1}{|\Vk - \Vk '|^2} d^3 k' 	&= c\int\limits_{0}^{\infty}\int\limits_{0}^{\pi}\int\limits_{0}^{2\pi} \Psi(k ')\frac{k '^2 \sin(\vartheta)}{k ^2 + k '^2 -2kk '\cos(\vartheta)} \,d\varphi \, d\vartheta \, dk' \\
									&& &= 2\pi c\int\limits_{0}^{\infty}\int\limits_{-1}^{1} \Psi(k ')\frac{k '^2}{k ^2 + k '^2 -2kk 'u} \,du \, dk' \\
									&& &= -2\pi c\int\limits_{0}^{\infty} \Psi(k ')\frac{k '^2}{2kk '}\ln{(k ^2 + k '^2 -2kk 'u)}\Big |_{\text{-}1}^1 \,dk' \\
									&& &= \underbrace{2\pi \,c}_{-\dfrac{e^2}{4\pi^2 \epsilon_0}} \int\limits_{0}^{\infty} \Psi(k ')\frac{k '}{k}\ln{\left |\frac{k + k '}{k-k'}\right |} \, dk' \\
\end{comment}

An dieser Stelle ergeben sich zwei Schwierigkeiten. Zum einen wird das Integral für $k = 0$ singulär, was sich jedoch als harmlos erweist, wenn man den Grenzwert $k \rightarrow 0$ betrachtet. Zum anderen divergiert der Logarithmus im Integral $\forall k=k'$, was einen zunächst vor eine Herausforderung stellt. \\ \\
\subsubsection{Punkt 1: $k\rightarrow0$}
\begin{alignat*}{3}
\lim_{k\rightarrow 0}\,\frac{k '}{k}\ln{\left |\frac{k + k '}{k-k'}\right |} \ \  &&  \stackrel{\text{L.H.}}{=} \ \ \lim_{k\rightarrow 0}\,\frac{k '}{1}\frac{1}{\left |\frac{k + k '}{k-k'}\right |}\left |-\frac{2k'}{(k-k')^2}\right | \ \ = \ \  \lim_{k\rightarrow 0}\,\left |\frac{2k '^2}{{k^2 - k '^2}}\right | \ \ = \ \ 2 
\end{alignat*}
Damit ist gezeigt, dass der Integrand für $k \rightarrow 0$ einen endlichen Wert annimmt und damit nicht zum Integral beiträgt. 
\subsubsection{Punkt 2: $k = k'$}
Die Lösung dieses Problems teilt sich auf in zwei wesentliche Überlegungen. Zum einen ist zweckmäßig das Integral so aufzuteilen, dass sich ein Teil ergibt, der nur für $k\neq k'$ Beiträge liefert. Entsprechend gibt es einen anderen Teil, der sämtliche Beiträge enthält, bei denen $k=k'$ gilt. Die zweite Überlegung folgt aus der Konvergenzbetrachtung der Integrale. Da diese über den gesamten Radialbereich ausgewertet werden, kann durch Verändern des Konvergenzverhaltens des Integranden der Grenzwert eines eigentlich divergierenden Integrals bestimmt werden. \\
Konkret ergibt sich: 
\begin{alignat*}{2}
\int\limits_{0}^{\infty} \Psi(k ')\underbrace{\frac{k '}{k}\ln{\left |\frac{k + k '}{k-k'}\right |}}_{\coloneqq \, V_\text{eff}(k,k')}\, dk' && = \int\limits_{0}^{\infty} \underbrace{\Psi(k ')\frac{k '}{k}\ln{\left |\frac{k + k '}{k-k'}\right |}-X}_{\stackrel{!}{=}0,\text{falls }k=k',\text{ Teil 1}}\, dk' + 
		 	\int\limits_{0}^{\infty} \underbrace{\Psi(k ')\frac{k '}{k}\ln{\left |\frac{k + k '}{k-k'}\right |}X}_{\text{Teil 2}}\, dk' 
\end{alignat*}
Es ist sofort ersichtlich, dass $X \propto \Psi(k)$ sein sollte, um die an Teil 1 gestellte Bedingung zu erfüllen. Bezüglich der Lösbarkeit von Teil 2 muss außerdem ein Konvergenzfaktor berücksichtigt werden, der die Form $g(k,k') = \frac{4k^4}{(k^2 + k'^2)^2}$ besitzt. So ergibt sich insgesamt für $X$: 
\begin{alignat*}{4}
&& X = g(k,k')\frac{V_\text{eff}(k,k')}{\Psi(k')}\Psi(k)  &&\ \ =\ \ && &\frac{4k^4}{(k^2+k '^2) ^2}\frac{V_\text{eff}(k,k')}{\Psi(k')}\Psi(k) \\
\text{Teil 1:}\qquad 	&&  \int\limits_{0}^{\infty} \Psi(k ')V_\text{eff}(k,k')-X \,dk' && \ \ = \ \ &&
											&\int\limits_{0}^{\infty} V_\text{eff}(k,k')\left [\Psi(k ')-\frac{4k^4}{(k^2+k '^2)^2}\Psi(k)\right ]\,dk'  \\ 
\text{Teil 2:}\qquad 	&& 	\int\limits_{0}^{\infty} \Psi(k ')V_\text{eff}(k,k')X   && \ \ = \ \ &&
											&\Psi(k)\int\limits_{0}^{\infty} \frac{4k^4}{(k^2+k '^2)^2}V_\text{eff}(k,k')\,dk'
\end{alignat*}
In Hinblick auf die numerische Behandlung muss Teil 2 noch weiter analysiert werden, während Teil 1 schon diskretisiert werden kann. Dazu löst man den Betrag im Logarithmus auf und integriert partiell, sodass sich ein einfaches Integral bildet, welches sich aufteilt in Logarithmen und einen Arkustangens.
\begin{alignat*}{2}
\int\limits_{0}^{\infty} \frac{4k^4}{(k^2+k '^2)^2}V_\text{eff}(k,k')\,dk' 
						&& \ \ &=\ \ \int\limits_{0}^{\infty} \frac{4k^4}{(k^2+k '^2)^2}\frac{k '}{k}\ln{\left |\frac{k + k '}{k-k'}\right |}\,dk' \\
%
%						Hier kommen Gleichungen aus untenstehendem Kommmentar rein
%
						&& \ \ &=\ \ 	k\Big[\underbrace{\ln\left (\frac{1+u}{1-u}\right )}_{\rightarrow 0 \text{ für } u\rightarrow \infty}+\, 2\arctan{u}\Big]_{0}^{\infty}\\  
						&& \ \ &=\ \ 	\pi k
\end{alignat*}
\begin{comment}
%
%					Diese Zeilen sind die explizite Rechnung des Integrals oben
%
						&& \ \ &=\ \ 	k\int\limits_{0}^{\infty} \frac{4u}{(1+u^2)^2}\ln{\left |\frac{1 + u}{1-u}\right |}\,du \qquad\text{Betrag auflösen}\\
						&& \ \ &=\ \ 	k\int\limits_{0}^{1} \left (\frac{-2}{1+u^2}\right )'\ln{\left (\frac{1 + u}{1-u}\right )}\,du + k\int\limits_{1}^{\infty} \left( \frac{-2}{1+u^2}\right )'\ln{\left (\frac{1 + u}{u-1}\right )}\,du\\
						&& \ \ &\stackrel{\text{P.I.}}{=}\ \ \underbrace{k\left (\frac{-2}{1+u^2}\right )\ln{\left ( \frac{1 + u}{1-u}\right )}\Big |_{0}^{1}+k\left (\frac{-2}{1+u^2}\right )\ln{\left ( \frac{1 + u}{u-1}\right )}\Big |_{1}^{\infty}}_{\text{C.H.}\,\rightarrow \,0}
						+k\int\limits_{0}^{\infty} \frac{2}{1+u^2}\frac{1 - u}{1+u}\frac{2}{(1-u)^2}\,du \\
						&& \ \ &=\ \ 	k\int\limits_{0}^{\infty} \frac{4}{(1-u^4)}\,du  \\
						&& \ \ &=\ \ 	k\int\limits_{0}^{\infty} \left (\frac{1}{1-u}+\frac{1}{1+u}+\frac{2}{1+u^2}\right )\,du\\
						&& \ \ &=\ \ 	k\Big[-\ln{(1-u)}+\ln{(1+u)}+ 2\arctan{u}\Big]_{0}^{\infty}\\  
%
\end{comment}

Damit ergibt sich nun: 
\begin{alignat*}{1}
	&-\dfrac{e^2}{4\pi ^2 \epsilon_0} \Big [\int\limits_{0}^{\infty} V_\text{eff}(k,k')\left [\Psi(k ')-g(k,k')\Psi(k)\right ]\,dk'  
+  \Psi(k)\int\limits_{0}^{\infty} g(k,k')V_\text{eff}(k,k')\,dk' \Big]   \\
= 	\qquad &-\dfrac{e^2}{4\pi^2 \epsilon_0} \int\limits_{0}^{\infty} V_\text{eff}(k,k')\left [\Psi(k ')-g(k,k')\Psi(k)\right ]\,dk' 
-  \dfrac{e^2}{4\pi \epsilon_0}  k\Psi(k) 
\end{alignat*}
Die Diskretisierung des Problems hat somit die Form: 
\begin{alignat*}{3}
\sum\limits_{k'}H_{kk'}\Psi_{k'} = E \Psi_k && \qquad \text{  wobei}\qquad && &H_{kk'}= 
	 \begin{cases}
     V_{kk'}  &k \neq k' \\
     \frac{\hbar ^2 k^2}{2\mu} -\frac{e^2}{4\pi \epsilon_0}k +\sum\limits_{k\neq k'}\frac{4k^4}{(k^2+k '^2)^2}V_{kk'} &k=k'
   \end{cases}\\
  &&  && &V_{kk'} = V_\text{eff}(k,k')=\frac{k '}{k}\ln{\left |\frac{k + k '}{k-k'}\right |}
\end{alignat*}
\subsection{Lösen des 2D Problems}
Das Vorgehen beim 2D Problem ist völlig analog zur Lösung in 3 Dimensionen. Einzig unterschiedlich ist die Lösbarkeit der Integrale, welche nur numerisch ausgewertet werden können. Angefangen wird erneut mit der Fouriertransformation des Coulombpotentials, welche die folgende Form besitzt: 
\begin{alignat*}{1} 
\mathcal{F}[V(r)](k)=-\frac{e^2}{2\varepsilon_0 V}\frac{1}{|\Vk|}\eqqcolon \widetilde{V}(k)
\end{alignat*}
Natürlich gilt für die Fouriertransformation der Schrödigergleichung eines isotropen Problems: 
\begin{alignat*}{2}
\left [-\frac{\hbar^2}{2\mu}\nabla ^2 + V(r)\right ]\Psi(\vec{r}) &= E\, \Psi(\vec{r}) \ \ \leftrightarrow \ \ \left [\frac{\hbar^2k^2}{2\mu}-E  \right ]\Psi(k) = \frac{V}{4\pi ^2} \int  \widetilde{V}(k-k')\Psi(k ') d^2 k'
\end{alignat*}
Das Faltungsintegral hat hierbei die Form: 
\begin{alignat*}{2}
\frac{V}{4\pi ^2} \int  \widetilde{V}(k-k')\Psi(\Vk ') d^2 k' 
		&& \ \ &= \ \ -\frac{e^2}{8\pi^2\varepsilon_0} \int  \frac{1}{|\Vk -\Vk '|}\Psi(k ') d^2 k' \\
		&& \ \ &= \ \ -\frac{e^2}{8\pi^2\varepsilon_0} \int\limits_{0}^{\infty}\int\limits_{0}^{2\pi}  \frac{k'}{\sqrt{k^2+k'^2-2kk'\cos(\varphi)}}\Psi(k ') d\varphi dk' \\
		&& \ \ &= \ \ \int\limits_{0}^{\infty}V_\text{eff}(k,k')\Psi(k ') dk' \\
\text{wobei}\qquad V_\text{eff}(k,k') &&  \ \ &= \ \ 		-\frac{e^2}{8\pi^2\varepsilon_0}\int\limits_{0}^{2\pi}  \frac{k'}{\sqrt{k^2+k'^2-2kk'\cos(\varphi)}} d\varphi
\end{alignat*}
Da das Integral über die Winkelkoordinate nun nicht mehr trivial zu lösen ist, hängt das effektive Potential $V_\text{eff}$ noch direkt von diesem ab. Wieder ergeben sich Schwierigkeiten an den Stellen, an denen der Integrand divergiert. Dies ist insbesondere der Fall, wenn $k=k'$ und $\varphi = 0$ gilt. Um dieses Problem zu lösen, wird der Faltungsausdruck erneut mithilfe des konvergenzerzeugenden Faktors $g(k,k')$  umgeschrieben: 
\begin{alignat*}{1}
g(k,k') \ \ &= \ \ \frac{4k^4}{(k^2+k '^2)^2} \\ 
\int\limits_{0}^{\infty}V_\text{eff}(k,k')\Psi(k ') dk' 
&=\underbrace{ \int\limits_{0}^{\infty}V_\text{eff}(k,k')\left [\Psi(k ')-g(k,k')\Psi(k)\right ] dk'}_{\text{Teil 1}} +\Psi(k) 
\underbrace{\int\limits_{0}^{\infty}g(k,k')V_\text{eff}(k,k')dk'}_{\text{Teil 2}}
\end{alignat*}
Es zeigt sich, dass der Integrand in Teil 2 dimensionslos dargestellt werden kann, woraus sich der Vorteil ergibt, dass dieses Integral unabhängig von $k$ oder $k'$ auf einem feinen $\varphi$-Gitter bestimmt werden kann. Es gilt: 
\begin{alignat*}{2}
\int\limits_{0}^{\infty}g(k,k')V_\text{eff}(k,k')dk' 
		&& \ \ &= \ \ \underbrace{-\frac{e^2}{8\pi^2\varepsilon_0}}_{\coloneqq \,c} 
		\int\limits_{0}^{\infty}\frac{4k^4}{(k^2+k '^2)^2}\int\limits_{0}^{2\pi}  \frac{k'}{\sqrt{k^2+k'^2-2kk'\cos(\varphi)}} d\varphi dk' \\
		&&\ \ &= \ \ c\cdot k \int\limits_{0}^{\infty}\int\limits_{0}^{2\pi} \frac{4u}{(1+u^2)^2\sqrt{1+u^2-2u\cos(\varphi)}} d\varphi du\\ 
		&&\ \ &= \ \ -\frac{e^2}{8\pi^2\varepsilon_0} k I
\end{alignat*}
Da $I$ konvergiert und damit numerisch zu bestimmen ist, folgt für die Diskretisierung des Problems die identische Darstellung wie in 3 Dimensionen:
\begin{alignat*}{3}
\sum\limits_{k'}H_{kk'}\Psi_{k'} = E \Psi_k && \qquad \text{  wobei}\qquad && &H_{kk'}= 
	 \begin{cases}
     V_{kk'}  &k \neq k' \\
     \frac{\hbar ^2 k^2}{2\mu} -\frac{e^2}{8\pi^2\varepsilon_0} k I +\sum\limits_{k\neq k'}\frac{4k^4}{(k^2+k '^2)^2}V_{kk'} &k=k'
   \end{cases}\\
  &&  && &V_{kk'} = V_\text{eff}(k,k')=-\frac{e^2}{8\pi^2\varepsilon_0}\int\limits_{0}^{2\pi}  \frac{k'}{\sqrt{k^2+k'^2-2kk'\cos(\varphi)}} d\varphi
\end{alignat*}


$x^2\Big |_a^b$
\end{document}
